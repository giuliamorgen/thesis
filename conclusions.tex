\chapter{Conclusioni e sviluppi futuri}

\section{Conclusioni}
Pur essendo nato nel 2011, il network Litecoin ha recentemente guadagnato valore ed attenzione da parte di investitori e sviluppatori. La crescita è stata esponenziale soprattutto negli ultimi due anni e l’interesse allo sviluppo software per la blockchain stessa e la sua analisi si sta solo ora avvicinando a quello di Bitcoin. Le sue caratteristiche lo rendono adatto a pagamenti elettronici e correntemente, dopo un tentativo con LitePay, la fondazione Litecoin sta valutando l’adozione di una carta di debito simile ad un bancomat per pagamenti “fisici”, per cui c’è da attendersi un incremento di interesse dal lato sia economico che software.
Seppur ancora distante dalla diffusione di Bitcoin si possono riscontrare in Litecoin diversi meccanismi comuni ad esso per via della similitudine tra i protocolli


\section{Sviluppi futuri}

Per quanto riguarda il tool, si può pensare sia ad un raffinamento delle analisi tramite adeguamento delle librerie già esistenti ai nuovi protocolli che all'implementazione di nuovi componenti per effettuare analisi analoghe a quelle che ho appena svolto su hard forks di Bitcoin come Bitcoin Cash. Quest’ultima seppur recente (nata il 1 agosto 2017) ha come obiettivo la riduzione delle fees e l’aumento della velocità di validazione delle transazioni sul network, ragioni per cui c’è da attendersi esiti e sviluppi simili a quelli di Litecoin.
Le tracce per il raffinamento delle analisi in un futuro prossimo sono tante: ad esempio il miglioramento delle API dei servizi web di terze parti, spesso ancora un po' carenti per Litecoin, e la maggior diffusione dei dati storici ad esso relativi permetterebbero analisi più accurate e significative riguardanti i bilanci, i tassi di scambio e le fees in relazione all'euro.

\subsection{Confronto con un lavoro correlato}
Il lavoro “Empirical Analysis of Crypto Currencies”\cite{relatedwork} mette a confronto gli ecosistemi Bitcoin e Litecoin ed esamina in ciascuno di essi il grado di correlazione tra transazioni input e output, e tra i top 100 indirizzi per ricchezza nelle due blockchain. Evidenzia che i più grossi tra i nodi sono miners e che in entrambi i casi i più ricchi tra di loro hanno scarsissima correlazione, il che evidenzia una maggior tendenza all’accumulo che alla spesa. Nel caso di Litecoin i $\frac{2}{3}$ degli indirizzi più ricchi risultano trasferire semplicemente i fondi da un account ad un altro tramite una piccola commissione.
L'autore misura inoltre il grado di clustering, ovvero i “triangoli” di transazioni a->b->a che possono indicare riciclaggio o semplici test di rete. Si tratta di un lavoro molto orientato alla statistica da cui si può trarre spunto per un'interpretazione in chiave statistica delle analisi già effettuate o di quelle future. 

