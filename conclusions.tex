\chapter{Conclusioni, lavori correlati e sviluppi futuri}

Pur essendo nato nel 2011, il network Litecoin ha recentemente guadagnato valore ed attenzione da parte di investitori e sviluppatori. La crescita è stata esponenziale soprattutto negli ultimi due anni e l’interesse allo sviluppo software per la blockchain stessa e la sua analisi si sta solo ora avvicinando a quello di Bitcoin. Le sue caratteristiche lo rendono adatto a pagamenti elettronici e correntemente, dopo un tentativo con LitePay, la fondazione Litecoin sta valutando l’adozione di una carta di debito simile ad un bancomat per pagamenti “fisici”, per cui c’è da attendersi un incremento di interesse dal lato sia economico che software. Seppur ancora distante dalla diffusione di Bitcoin si possono riscontrare in Litecoin diversi meccanismi comuni ad esso per via della similitudine tra i due protocolli.


La presente tesi studia i primi 1470000 blocchi della blockchain di Litecoin (tutti i blocchi disponibili sino al 07/08/2018),
focalizzandosi sui metadati salvati da applicazioni esterne e l'attività dei miner.
Tra i risultati principali è interessante osservare che, nonostante Litecoin sia decentralizzato, l'85\% dei blocchi di cui è noto il miner sono stati minati da soli 5 mining pool.
Inoltre, la tesi evidenzia una crescita del fenomeno del merged mining (dal 2016 la percentuale di blocchi interessati al fenomeno è superiore al 97\%) e una tendenza di alcuni mining pool a minare blocchi vuoti. Il 90\% delle transazioni OP\_RETURN utilizzano SegWit. La tesi presenta 10 protocolli che utilizzano regolarmente OP\_RETURN ma non utilizzano SegWit.


Il lavoro “Empirical Analysis of Crypto Currencies”\cite{relatedwork} mette a confronto gli ecosistemi Bitcoin e Litecoin ed esamina in ciascuno di essi il grado di correlazione tra transazioni input e output, e tra i 100 indirizzi più ricchi nelle due blockchain. Evidenzia che i nodi più grossi sono miner e che in entrambi i casi i più ricchi tra di loro hanno scarsissima correlazione, il che evidenzia una maggior tendenza all’accumulo che alla spesa. Nel caso di Litecoin i $\frac{2}{3}$ degli indirizzi più ricchi risultano trasferire semplicemente i fondi da un account ad un altro tramite una piccola commissione. \cite{relatedwork} misura inoltre il grado di clustering, ovvero i “triangoli” di transazioni a->b->a che possono indicare riciclaggio o semplici test di rete.
Mentre \cite{relatedwork} si concentra quindi sullo studio degli spostamenti di denaro tra gli indirizzi, la presente tesi analizza i metadati salvati da applicazioni esterne e l'attività dei miner (in particolar modo la distribuzione dei mining pool, i blocchi vuoti e il fenomeno del merged mining).


Relativamente agli sviluppi futuri, è interessante raffinare le analisi adeguando le librerie già esistenti (come LitecoinJ) alle versioni più recenti (ad esempio supportare SegWit). 
Inoltre, è rilevante estendere le analisi sviluppate nella tesi a ulteriori blockchain nate da hard fork di Bitcoin (ad esempio Bitcoin Cash). Quest’ultima, seppur recente (nata il 1 agosto 2017), ha come obiettivo la riduzione delle fee e l’aumento della velocità di validazione delle transazioni sul network, ragioni per cui c’è da attendersi esiti e sviluppi simili a quelli di Litecoin.
Ci sono numerosi aspetti che meritano ulteriori approfondimenti: ad esempio il miglioramento delle API dei servizi web di terze parti, utili all'ottenimento di informazioni più precise. Infine la maggior diffusione dei dati storici di Litecoin permetterebbe la realizzazione di analisi più accurate e significative riguardanti i bilanci, i tassi di scambio e le fees in relazione all'euro.

