\vspace{4cm}
La tesi ha come obiettivo l'analisi della blockchain Litecoin con i presupposti di un confronto con ciò che sappiamo di Bitcoin. Partendo da un background tecnico sul protocollo Bitcoin valido per tutte le blockchain che si basano su di esso, introduce Litecoin e le sue caratteristiche, con un'attenzione particolare ad analogie e differenze da Bitcoin, gli utilizzi attuali e le prospettive future. Contiene delle analisi svolte tramite il tool BlockAPI, fino ad ora già operativo per le blockchain Bitcoin ed Ethereum, che ho esteso a Litecoin implementando i componenti necessari. Le analisi presenti nella tesi sono relative alla distribuzione dell’hashing power tra i mining pools, il mining di blocchi vuoti, la diffusione del merged mining e l’utilizzo dell’OP\_RETURN per segnalare l’adesione a protocolli o servizi. Ciascuna analisi è accompagnata da qualche riga di codice significativa per il suo svolgimento ed i suoi risultati vengono presentati e contestualizzati.